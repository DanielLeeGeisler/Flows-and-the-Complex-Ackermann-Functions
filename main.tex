\documentclass{article}
\usepackage{maa-monthly}

%% IF YOU HAVE FONTS INSTALLED
%\usepackage{mtpro2}
%\usepackage{mathtime}

%\theoremstyle{theorem}
\newtheorem{theorem}{Theorem}
\newtheorem{proposition}[theorem]{Proposition}
\newtheorem{lemma}[theorem]{Lemma}
\newtheorem{corollary}[theorem]{Corollary}

\theoremstyle{definition}
\newtheorem*{definition}{Definition}
\newtheorem*{remark}{Remark}

\begin{document}

\title{Flows and the Complex Ackermann Function}
\markright{Complex Ackermann Function}
\author{Author}

\maketitle

\begin{abstract}
The process of extending maps to flows is called continuous or fractional iteration. Extending tetration to the complex numbers is a classic example, finding functional square roots is another.

A general method of fractional iteration in the complex plane is presented based on Faà di Bruno's formula which decomposes the iterations of holomorphic functions into recursive Bell polynomials.

The classical Ackermann function along with fractional iteration defines an Ackermann function for complex numbers. 
\end{abstract}

\noindent

Consider the holomorphic function $f(z): \mathbb{C} \rightarrow \mathbb{C}$ and its iterates $f^t(z), t \in \mathbb{N}$ with a fixed point $L\in\mathbb{C}$ such that $f(L)=L$. The derivative of iterated function $D^nf^t(L)$ is $Df^t(L)=f'(L)^t$ and the most important factor in the dynamics of the map.

\begin{theorem}[Recursive Bell Polynomial Theorem]
The derivatives of an iterated holomorphic function with a non-superattracting fixed point can be expressed in terms of recursive Bell polynomials.
$$D^nf^t(L)=\sum_{r=0}^\infty(\sum_{k=2}^n \frac{f^{(k)}(L)}{k!} B_{n,k}(D^2f^{t-1}(L),\ldots, D^{n-k+1}f^{t-1}(L)))^r$$
\end{theorem}

\begin{proof}
Consider the following version of Faà di Bruno's formula, \cite{weisstein}

${D^n} f(g(x)) = \sum_{k=1}^n f^{(k)}(g(x))\cdot B_{n,k}\left(g'(x),g''(x),\dots,g^{(n-k+1)}(x)\right)$.

Set $g(x)=f^{t-1}(x)$ and $x=L$,

$D^nf^t(L)=\sum_{k=1}^n \frac{f^{(k)}(L)}{k!} B_{n,k}(Df^{t-1}(L),\ldots, D^{n-k+1}f^{t-1}(L))$

By cases. There are two cases: $k<n$ and $k=n$.

Case 1. ($k<n$). $D^kf^{t-1}(L)$ is already known.

Case 2. ($k=n$). $D^nf^{t-1}(L)$ is not known but can be solved by a simple recursive expression resulting in the geometrical series where $C$ is a constant

$D^nf^t(x)=C+f'(L)D^nf^{t-1}(x)$.

By Taylor series $f^t(x)=\sum_{k=0}^\infty\frac{1}{k!} D^nf^t(L) (x - L)^k$
\end{proof}

While initially $t \in \mathbb{N}$, once a symmetry is added the result is consistent with $t \in \mathbb{C}$.

Mathematica code
\begin{verbatim}
   Flow[f_, t_, x_, L_, order_ : 3] := Module[{s},
   H[0] = L;
   H[1] = f'[L]^t ;
   Do[H[max] = First[r[t] /. 
      RSolve[{r[0] == 0, r[t] == Sum[Derivative[k][f][L] BellY[max, k,Table[H[j] /. t -> t - 1, {j, max}]], {k, 2, max}] + f'[L] r[t - 1]}, r[t], t]], 
   {max, 2, order}];
   s = Sum[1/k! H[k] (x - L)^k, {k, 0, order}]
];
\end{verbatim}

\begin{acknowledgment}{Acknowledgment.}
The authors wish to thank the Greek polymath Anonymous, whose prolific works are an endless source of inspiration.
\end{acknowledgment}

\begin{thebibliography}{1}
\bibitem{weisstein} Weisstein, Eric W. "Faà di Bruno's Formula." From \textit{MathWorld}--A Wolfram Web Resource. https://mathworld.wolfram.com/FaadiBrunosFormula.html

\end{thebibliography}


\vfill\eject

\end{document}
